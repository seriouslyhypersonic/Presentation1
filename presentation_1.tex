% ---------------------------------------------------------------------------- %
%																		       %
% 	         Title: Introduction to the Adaptive framework					   %
%           Author: Nuno Borges Loureiro Alves de Sousa					       %
%	          Date: February 2019		       							       %
%																		       %
%           --------------------------------------------------                 %
%																		       %
%			E-mail: nunoalvesdesousa@me.com (main)						       %
%					nunoalvesdesousa@tecnico.ulisbboa.pt				       %
%			 Phone: (+351) 915 547 571									       %
%																		       %
% ---------------------------------------------------------------------------- %

% ---------------------------------------------------------------------------- %
% Preamble
% ---------------------------------------------------------------------------- %

\documentclass{beamer}

\usepackage[T1]{fontenc}
\usepackage[utf8]{inputenc}
\usepackage{lmodern}

\usepackage[whitebg, frametitlelogo]{beamerthemeLisboa}

% Quotations
\usepackage{csquotes}
\usepackage{attrib}

% Custom command for descriptions
\newcommand{\dindent}{\hspace{15pt}} 		 % default indentation
\newcommand{\bitem}[1][]{\item[\textbf{#1}]} % bold item

% Enumeration symbols
\newcommand{\itemd}{\item[-]}

% Code listings
\usepackage{sourcecodepro} % Nice monospaced font
\usepackage{relsize}	   % More granular control over font size

% Light theme
\definecolor{PreprocessorColor}{RGB}{49, 206, 126}
\colorlet{DarkGreen}{green!50!black}            % Comment color
\colorlet{FunctionColor}{cyan}                  % Comment color
\definecolor{MemberColor}{rgb}{0.6, 0.4, 0.8}   % Member color
\colorlet{MacroColor}{yellow!70!red}            % Macro color

\usepackage{listings}

% Temporary fix -------------------------------------------------------------- %
%\usepackage{lstlinebgrd} currently norworking
% https://tex.stackexchange.com/questions/451532/recent-issues-with-lstlinebgrd-package-with-listings-after-the-latters-updates
\makeatletter
\let\old@lstKV@SwitchCases\lstKV@SwitchCases
\def\lstKV@SwitchCases#1#2#3{}
\makeatother
\usepackage{lstlinebgrd}
\makeatletter
\let\lstKV@SwitchCases\old@lstKV@SwitchCases

\lst@Key{numbers}{none}{%
	\def\lst@PlaceNumber{\lst@linebgrd}%
	\lstKV@SwitchCases{#1}%
	{none:\\%
		left:\def\lst@PlaceNumber{\llap{\normalfont
				\lst@numberstyle{\thelstnumber}\kern\lst@numbersep}\lst@linebgrd}\\%
		right:\def\lst@PlaceNumber{\rlap{\normalfont
				\kern\linewidth \kern\lst@numbersep
				\lst@numberstyle{\thelstnumber}}\lst@linebgrd}%
	}{\PackageError{Listings}{Numbers #1 unknown}\@ehc}}
\makeatother
% ---------------------------------------------------------------------------- %

% Dimensions
\lstset{
	numbers     = left, linebackgroundheight=0.90\baselineskip,
	numbersep   = 10pt,
	numberstyle = \tiny,
	xleftmargin = 0.06\textwidth, xrightmargin=0.06\textwidth
}

\lstdefinestyle{myCpp}
{
	language            = [ISO]C++,
	basicstyle          = \ttfamily\scriptsize\relscale{0.90},
	showstringspaces    = false,
	keywordstyle        = \bfseries\color{blue},
	stringstyle         = \color{red},
	commentstyle        = \itshape\color{DarkGreen},
    % MKL keywords
	emph                = {CblasColMajor,CblasTrans}, 
    emphstyle           = \color{MemberColor},
    % Preprocessor directives
    directivestyle      = \bfseries\color{PreprocessorColor},
    % Overloaded operators
	literate            = {oveq}{{{\bfseries\color{magenta}==}}}2
						  {ov=}{{{\bfseries\color{magenta}=}}}1
						  {ov+}{{{\bfseries\color{magenta}+}}}1
	                      {ov-}{{{\bfseries\color{magenta}-}}}1
	                      {ov*}{{{\bfseries\color{magenta}*}}}1,
	keywordstyle        = [2]{\color{brown}},
	keywordstyle        = [3]{\color{MacroColor}},
	otherkeywords       = {\&, ;, ::, static_assert, decltype, REQUIREMENT, REQUIRES, CONCEPT},
	morekeywords        = [2]{;, ::},
	morekeywords        = [3]{REQUIREMENT, REQUIRES, CONCEPT},
    linebackgroundcolor = {\ifodd\value{lstnumber}\color{istblue!10}
                           \else\color{gray!5}\fi},
    frame               = tb,
}

\definecolor{DarkBackground}{RGB}{66, 72, 78}
\definecolor{KeywordColor}{RGB}{250, 140, 213}

\lstset{
	style = myCpp
}

% Words as code
\newcommand{\code}[1]{{\scriptsize\texttt{#1}}}

% Centered block with variable width
\usepackage{tikz}
\usepackage{pgfplots}

% Hyper ref
\usepackage{hyperref}
\newcommand{\linkstyle}[1]{\textcolor{cyan}{\underline{#1}}}

% ---------------------------------------------------------------------------- %
% Outline
% ---------------------------------------------------------------------------- %

\AtBeginSection[]{
	\frame{
		\frametitle{Outline}
		\tableofcontents[current, hideallsubsections]
	}
}

\AtBeginSubsection[]{
	\frame{
		\frametitle{Outline}
		\tableofcontents[currentsection, currentsubsection]
	}
}

% ---------------------------------------------------------------------------- %
% Cover
% ---------------------------------------------------------------------------- %

\title{The Adaptiv Framework}
\author{Nuno Alves de Sousa}
\institute{
	Instituto Superior Técnico\\
	\tiny Área Científica de Mecânica Aplicada e Aeroespacial
}
\date{\today}

\begin{document}
	
\begin{frame}[plain]
	\vspace{2.0cm}
	\centering
	\includegraphics[keepaspectratio=true, height=2cm]{figures/adaptive.png}
	\vspace{0.75cm}
	\titlepage
\end{frame}

% ---------------------------------------------------------------------------- %
% Slides
% ---------------------------------------------------------------------------- %

\section{Best coding practices}
\subsection{Software quality}

\begin{frame}{Attributes of good software}
	\begin{quote}
		\enquote{Software and cathedrals are much the same – first we build them, then we pray.}\\
		\attrib{Anonymous}
	\end{quote}

%	\begin{quote}
%		\enquote{Without requirements or design, programming is the art of adding bugs to an empty text file.}\\
%		\attrib{Louis Srygley}
%	\end{quote}
\end{frame}

\begin{frame}{Attributes of good software}
	Not \structure{\textit{what}} the program does, but \structure{\textit{how well}} it does it:
	\vspace{1\baselineskip}
	\begin{description}[\dindent Maintainability]
		\bitem[Maintainability] reduce/reverse \enquote{code entropy} \\
								cheaper/safer to change than to rewrite
		\bitem[Dependability] availability, reliability, safety, integrity
		\bitem[Efficiency] algorithmic efficiency\\
						   storage efficiency
		\bitem[Usability] \enquote{consumer} effectiveness and efficiency\\
						  elegance and clarity perceived by the user
	\end{description}
\end{frame}

\subsection{Prerequisites}
\begin{frame}{Where to start?}
	\begin{quote}
		\enquote{What happens before one gets to the coding stage is often of crucial importance to the success of the project.}
		\attrib{Meek \& Heath - Guide to Good Programming Practice}
	\end{quote}
	\vspace{1\baselineskip}
	Higher-level prerequisites to provide a solid foundation for coding:
	\begin{itemize}
		\item Coding standards
		\item Choice of programming language
		\item Life cycle, architecture, design
		\item \alert{Requirements}
	\end{itemize}
\end{frame}

\begin{frame}{Coding standards}
	Coding conventions are particularly important in collaborative projects:
	\begin{itemize}
		\item Much easier to read someone else's code
		\item Uniform style (\textit{e.g.} naming conventions for filenames, variables, etc)
		\item Deal with undereducated programmers
		\item Avoid insufficient library use
		\item Portability
		\item Commenting conventions:
			\begin{itemize}
				\itemd Speed up knowledge transfer
				\itemd Comment only what code expresses poorly (intent)
				\itemd Comments lie, code never lies
				\itemd Do not comment code modifications (use a \structure{version control system})
			\end{itemize}
	\end{itemize}
\end{frame}

\begin{frame}{Version control}
	\structure{Source code} is the most valuable asset of any software project
	\begin{block}{Version control systems (VCS)}
		\begin{itemize}
			\item Management of changes to all non-binary files
			\item Complete retrace of all versions of each file
			\item History of the authors of such changes
		\end{itemize}
	\end{block} 
	
	\begin{exampleblock}{Critical advantages}
		\begin{itemize}
			\item Rollback of all tracked changes
			\item Work in an isolated fashion
			\item Seamless team collaboration
			\item Efficient and flexible scaleability
		\end{itemize}
	\end{exampleblock}
\end{frame}

\begin{frame}{git - the world's leading version control system}
	\begin{columns}[onlytextwidth]
		\column{0.5\linewidth}
		\begin{center}
			\hspace{-1.7cm}
			\includegraphics[keepaspectratio=true, width=0.45\linewidth]{figures/git.png}
		\end{center}
		Why git?
		\begin{itemize}
			\item Free and open-source
			\item Small and fast
			\item Encourages branching
			\item Distributed
			\item Built-in IDE support
		\end{itemize}
		As a service:
		\begin{itemize}
			\item Source code hosting
			\item Code sharing platform
			\item GitHub, GitLab, etc.
		\end{itemize}
		\column{0.5\linewidth}
		\includegraphics[keepaspectratio=true, width=0.95\linewidth]{figures/github.png}
	\end{columns}
\end{frame}

%\begin{frame}{Choice of programming language}
%	C++ (hard, lack of knowledge, modern features)\\
%	Use good well tested libraries (boost) - portability
%\end{frame}

%\begin{frame}{WIP}
%	Life cycle, architecture, design all depend on the \alert{requirements}
%\end{frame}

\subsection{Code development}
\begin{frame}{Build system}
	\begin{columns}[onlytextwidth]
		\column{0.65\linewidth}
		\begin{itemize}
			\item Open-source, cross-platform set of tools to build, test and package software.
			\item Controls compilation process using platform and compiler independent config. files
		\end{itemize}
		\column{0.30\linewidth}
		\includegraphics[keepaspectratio=true, width=1\linewidth]{figures/cmake.png}
	\end{columns}
	\vspace{1\baselineskip}
	\begin{quote}
		\centering
		\structure{The defacto standard for building C++ projects}
	\end{quote}
	\begin{exampleblock}{Advantages}
		\begin{itemize}
			\item More time for coding
			\item Supported by most popular IDEs (\textit{e.g.} VS, JetBrains, QtCreator)
			\item Support for multiple compilers (\textit{e.g.} MSVC, GCC, Clang, Intel)
			\item Easy integration of 3rd party libraries
		\end{itemize}
	\end{exampleblock}	  
\end{frame}

\begin{frame}{Testing}
	\begin{columns}[onlytextwidth]
		\column{0.55\linewidth}
		\begin{quote}
			\enquote{Beware of bugs in the above code; I have only proved it correct, not tried it.}\\
			\attrib{Donald Knuth}
		\end{quote}
		\begin{itemize}
			\item Testing is essential when collaborating
			\item Can prevent reviewing bad code
			\item Should keep up with development
			\item Write once, test many
			\item Popular frameworks:
			\begin{itemize}
				\item[-] Google Test (on the right)
				\item[-] catch2 (header only)
			\end{itemize}
		\end{itemize}
		\column{0.45\linewidth}
		\includegraphics[keepaspectratio=true, width=1\linewidth]{figures/gtest.png}
	\end{columns}
\end{frame}

%\begin{frame}{Benchmarking}
%	content...
%\end{frame}

\section{Concepts Library}

\begin{frame}{C++ templates}
	What are templates?
	\begin{itemize}
		\item Foundation of generic programming
		\item Blueprint for creating a generic class or function
	\end{itemize}
	
	What are their uses?
	\begin{itemize}
		\item Avoid repeating code
		\item Generate code at compile-time
		\item Perform compile-time computations
	\end{itemize}

	But really... why bother?
	\begin{itemize}
		\item \structure{C++ template magic}!
		\begin{itemize}
			\item[-] Static polymorphism (no overhead)
			\item[-] Higher chances for compiler optimizations (\textit{e.g.} inlining)
			\item[-] Create elegant interfaces with highly optimized implementations\\
			\vspace{5pt}
			and more...
		\end{itemize}
	\end{itemize}
\end{frame}

\begin{frame}{C++ template metaprogramming (TMP)}
	Object-oriented programming and TMP techniques allow OpenFOAM users to represent
	\begin{equation}
		\frac{\partial}{\partial t}\left(\rho\mathbf{U}\right) + \nabla\cdot\left(\phi\mathbf{U}\right) - \mu\nabla^2\mathbf{U} = -\nabla p,
	\end{equation}
	with a syntax that closely resembles the mathematical formulation:
	\lstinputlisting{code/openfoam.cpp}
	Note: what if $\mathbf{U}$ is not actually a vector field?
\end{frame}

\begin{frame}{Metaprogramming pitfalls}
	Becoming a template wizard takes time (and a great deal of insanity):
	\begin{itemize}
		\item Many TMP techniques require knowledge of specific C++ idioms
		\item Frequently, error messages are cryptic:
		\begin{itemize}
			\item[-] Most errors are triggered only on template instantiation
			\item[-] Stack trace might be very deep
			\item[-] Type names can be extremely long (\textit{e.g.} templates instantiations as template arguments)
			\item[-] Overload resolution failure can produce a long list of candidates
		\end{itemize} 
	\end{itemize}

	Inexperienced programmers can easily get stuck (and frustrated) but...
	\begin{center}
		\structure{Often, TMP errors are related to instantiation with an invalid type}
	\end{center}
\end{frame}

\begin{frame}{C++ concepts}
	Concepts are constraints that limit the set of arguments accepted as template parameters:
	\begin{itemize}
		\item Type-checking
		\item Simplified compiler diagnostics
		\item Select overloads/specializations based on type properties (introspection)
	\end{itemize}
	\vspace{5pt}
	Concepts allows us to enforce an interface on a type without the overhead of inheritance.\\
	\begin{flushright}
		Example...
	\end{flushright}
\end{frame}

\begin{frame}{Custom concept using the concepts library}
	\lstinputlisting[basicstyle=\tiny]{code/concepts.cpp}
\end{frame}

\begin{frame}{Library summary}
	\begin{itemize}
		\item The concepts library is based on C++17
		\item Models all the future C++20 concepts in header \code{<concepts>}
		\begin{itemize}
			\item[-] Core language concepts (\textit{e.g.} \code{Same}, \code{DerivedFrom}, \code{ConvertibleTo}, ...)
			\item[-] Comparison concepts (\textit{e.g.} \code{Boolean}, \code{EqualityComparable}, ...)
			\item[-] Object concepts (\textit{e.g.} \code{Movable}, \code{Copyable}, ...)
			\item[-] Callable concepts (\textit{e.g.} \code{Invocable}, \code{Predicate}, ...)
		\end{itemize}
		\item Allows users to easily define new concepts
		\item Uses TMP techniques (SFINAE \& detection idiom)
		\item Introduces C++20 type traits ({\code{traits::common\_reference}})
	\end{itemize}
	\vspace{7pt}
	\begin{center}
		\scriptsize
		\linkstyle{\url{https://github.com/seriouslyhypersonic/experimental_concepts}}
	\end{center}
\end{frame}

\section{Linear Algebra Library}
\begin{frame}{CMatrix library (MDO GUI)}
Updates:
\begin{itemize}
	\item Build system changed to CMake
	\item Created FindMKL cmake module
	\item Works on Linux and Windows
	\item Does not work on macOS (library bug)
\end{itemize}

Issues:
\begin{itemize}
	\item Probably pre-C++11
	\item Inefficient:
	\begin{itemize}
		\item[-] No support for sparse matrices (?)
		\item[-] Does not use rvalue references (unnecessary temporaries)
		\item[-] Does not meet MKL memory alignment requirements (SSE, AVX)
		\item[-] Eager evaluation generates unoptimized code
	\end{itemize}
	\item Interface is complex and lacks uniformity
\end{itemize}
	\begin{center}
		\scriptsize
		\linkstyle{\url{https://github.com/seriouslyhypersonic/CMatrix}}
	\end{center}
\end{frame}

\begin{frame}[fragile]{Interface elegance \textit{vs} code efficiency}
	Level 3 BLAS operations $T(n) = O(n^3)$\\
	Example:
	\begin{equation}
		C \leftarrow \alpha A^T B^T + \beta C
	\end{equation}
	Desired interface:
	\lstinputlisting{code/dgemm.cpp}
\end{frame}

\begin{frame}[fragile]{Interface elegance \textit{vs} code efficiency}
	For an efficient implementation, the statement
	\lstinputlisting[linerange=12-12, firstnumber=12]{code/dgemm.cpp}
	should be translated into a call to the specialized CBLAS function:
	\lstinputlisting[linerange=12-18, firstnumber=12]{code/cblas_dgemm.cpp}
	Overhead: \\
	\begin{center}
		\textcolor{DarkGreen}{\textbf{1}} function call\\
		\textcolor{DarkGreen}{\textbf{0}} temporaries
	\end{center}
\end{frame}

\begin{frame}{Conventional operator overloading}
	Due to the normal order of evaluation of the C++ language,
	\lstinputlisting[linerange=12-12, firstnumber=12]{code/dgemm.cpp}
	
	leads to the following execution context:
	\lstinputlisting[firstline=9, firstnumber=9, escapechar=@]{code/operator_dgemm.cpp}
\end{frame}

\begin{frame}{Conventional operator overloading}
	Overhead: \\
	\begin{center}
		\textcolor{red}{\textbf{12}} function calls\\
		\textcolor{red}{\textbf{6}} temporaries
	\end{center}
\end{frame}

\begin{frame}{Performance comparison on a \small E5620$_{(\small76.8\,\text{GFLOPS, SSE4.2})}$} 
	% This file was created by matlab2tikz.
%
%The latest updates can be retrieved from
%  http://www.mathworks.com/matlabcentral/fileexchange/22022-matlab2tikz-matlab2tikz
%where you can also make suggestions and rate matlab2tikz.
%
\definecolor{mycolor1}{rgb}{0.00000,0.44700,0.74100}%
\definecolor{mycolor2}{rgb}{0.85000,0.32500,0.09800}%
\pgfplotsset{compat=1.13}
%
\begin{tikzpicture}

\begin{axis}[%
	width=0.8\textwidth,
	height=0.60\textheight,
	% scaling
	scale only axis,
	% x-axis
	xmode=log,
	xmin=2,
	xmax=1e4,
	xminorticks=true,
	xlabel={Matrix dim (n x n)},
	% y-axis
	ymin=0,
	ymax=9,
	ylabel={GFLOPS},
	% title
	title={$C \leftarrow \alpha A^T B^T + \beta C$},
	% grid
	xmajorgrids, xminorgrids,
	ymajorgrids,
	ylabel shift=11pt,
	% legend
	legend pos= north west,
	legend style={legend cell align=left, align=left, draw=white!15!black}
]
\addplot [color=istblue, line width=1pt]
  table[row sep=crcr]{%
2	0.0571428571428571\\
3	0.13695652173913\\
4	0.31304347826087\\
5	0.387323943661972\\
6	0.9\\
7	0.954545454545455\\
8	1.37721518987342\\
9	1.27190082644628\\
10	1.7948717948718\\
20	5\\
30	5.90322580645161\\
40	0.577308566083122\\
50	0.724616885725765\\
60	1.04638593288333\\
70	1.20609594302074\\
80	1.81449979748886\\
90	1.8049417065754\\
100	2.12195430936194\\
200	5.07100674026582\\
300	7.00851024131354\\
400	6.91090667903325\\
500	7.63949519803159\\
600	8.09153790873158\\
700	6.430759430825\\
800	8.14622240242962\\
900	8.16624589255426\\
1000	8.379046103597\\
2000	7.89342592638261\\
3000	8.09686433715722\\
4000	7.96154061433023\\
5000	7.56617471274107\\
6000	8.06676936003361\\
7000	7.88758805357199\\
8000	7.95636702664906\\
9000	8.07318059000709\\
};
\addlegendentry{\scriptsize MKL}

\addplot [color=red, line width=1pt]
  table[row sep=crcr]{%
2	0.0163934426229508\\
3	0.0431506849315069\\
4	0.0712871287128713\\
5	0.115546218487395\\
6	0.1828125\\
7	0.250853242320819\\
8	0.302222222222222\\
9	0.355427251732102\\
10	0.456521739130435\\
20	1.36439267886855\\
30	2.02583025830258\\
40	0.710448415743888\\
50	0.836563628532618\\
60	0.99563438549976\\
70	1.24208974543363\\
80	1.55136331471416\\
90	1.68453345282824\\
100	1.86407983084171\\
200	3.92260437453535\\
300	4.9745707374946\\
400	5.47014383883222\\
500	5.68158598549691\\
600	6.27797872201402\\
700	6.00020977003959\\
800	6.47883049218473\\
900	6.47557029283688\\
1000	6.97137243015563\\
2000	7.6068615754626\\
3000	7.74710213425786\\
4000	6.97452437509534\\
5000	7.05482177069607\\
6000	7.41447039864903\\
7000	7.59596977329975\\
8000	7.33438854073411\\
9000	7.66068416932429\\
};
\addlegendentry{\scriptsize CMatrix}

\end{axis}
\end{tikzpicture}%
\end{frame}

\begin{frame}{Performance comparison on a \small i7-4770k$_{(\small224\,\text{GFLOPS, AVX2})}$} 
	\input{figures/dgemm_i7.tex}
\end{frame}

\begin{frame}{Is there a solution?}
	The problem is that the compiler is too eager when evaluating the rhs of
	\lstinputlisting[linerange=12-12, firstnumber=12]{code/dgemm.cpp}
	What do we need?
	\begin{itemize}
		\item Bypass the normal order of evaluation of the C++ language
		\item An expression now represents a computation \structure{at compile time}
		\item The result is a structure representing that computation
		\item Expressions are evaluated only as needed (lazy evaluation)
	\end{itemize}
	What will the structure look like?
	\lstinputlisting[linerange=12-17, firstnumber=12]{code/expression_templates.cpp}
\end{frame}

\begin{frame}{How is it done?}
	We need \structure{logic} to instruct the \structure{compiler} how to \structure{build} the structure
	\begin{center}
		$\Rightarrow$ we need TMP!\\
		
		(Because we're using TMP, the structure is really just a type)
	\end{center}
	This technique is know as \structure{expression templates} (ET):
	\begin{itemize}
		\item We'll abuse C++'s type system
		\item Use expressions to build a structure (type) during compile time
		\item Use that type to perform optimizations
		\item Only evaluate when needed (assignment triggers evaluation)
	\end{itemize}
\end{frame}

\begin{frame}{Expression templates}
	The machinery behind expression templates is complex:
	\begin{itemize}
		\item Curiously recurring template pattern
		\item Recursing down the expression tree must be done efficiently
		\item Different type of optimizations:
		\begin{itemize}
			\item[-] Recursive nature (\textit{e.g.} loop fusion)
			\item[-] Specializations (\textit{e.g.} calls to MKL functions)
		\end{itemize}
	\end{itemize}

	Once implemented:
	\begin{itemize}
		\item Elegant interface generates efficient code
		\item ET can be used to accelerate automatic differentiation
		\item Will be possible to embed DSLs (\textit{à la} OpenFOAM)
		\item Optimizations and parsing happens at compile time
	\end{itemize}
\end{frame}

\end{document}