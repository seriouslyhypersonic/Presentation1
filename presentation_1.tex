% ---------------------------------------------------------------------------- %
%																		       %
% 	         Title: Introduction to the Adaptive framework					   %
%           Author: Nuno Borges Loureiro Alves de Sousa					       %
%	          Date: February 2019		       							       %
%																		       %
%           --------------------------------------------------                 %
%																		       %
%			E-mail: nunoalvesdesousa@me.com (main)						       %
%					nunoalvesdesousa@tecnico.ulisbboa.pt				       %
%			 Phone: (+351) 915 547 571									       %
%																		       %
% ---------------------------------------------------------------------------- %

% ---------------------------------------------------------------------------- %
% Preamble
% ---------------------------------------------------------------------------- %

\documentclass{beamer}

\usepackage[T1]{fontenc}
\usepackage[utf8]{inputenc}
\usepackage{lmodern}

\usepackage[whitebg, frametitlelogo]{beamerthemeLisboa}

% Quotations
\usepackage{csquotes}
\usepackage{attrib}

% Custom command for descriptions
\newcommand{\dindent}{\hspace{15pt}} 		 % default indentation
\newcommand{\bitem}[1][]{\item[\textbf{#1}]} % bold item

% Enumeration symbols
\newcommand{\itemd}{\item[-]}

% Code listings
\usepackage{sourcecodepro} % Nice monospaced font
\usepackage{relsize}	   % More granular control over font size

% Light theme
\definecolor{PreprocessorColor}{RGB}{49, 206, 126}
\colorlet{DarkGreen}{green!50!black}            % Comment color
\colorlet{FunctionColor}{cyan}                  % Comment color
\definecolor{MemberColor}{rgb}{0.6, 0.4, 0.8}   % Member color

\usepackage{listings}
\usepackage{lstlinebgrd}

% Dimensions
\lstset{
	numbers     = left, linebackgroundheight=0.90\baselineskip,
	numbersep   = 10pt,
	numberstyle = \tiny,
	xleftmargin = 0.06\textwidth, xrightmargin=0.06\textwidth
}

\lstdefinestyle{myCpp}
{
	language            = [ISO]C++,
	basicstyle          = \ttfamily\scriptsize\relscale{0.90},
	showstringspaces    = false,
	keywordstyle        = \bfseries\color{blue},
	stringstyle         = \color{red},
	commentstyle        = \itshape\color{DarkGreen},
    % MKL keywords
	emph                = {CblasColMajor,CblasTrans}, 
    emphstyle           = \color{MemberColor},
    % Preprocessor directives
    directivestyle      = \bfseries\color{PreprocessorColor},
    % Overloaded operators
	literate            = {ov=}{{{\bfseries\color{magenta}=}}}1
	                      {ov+}{{{\bfseries\color{magenta}+}}}1
	                      {ov*}{{{\bfseries\color{magenta}*}}}1,
	keywordstyle        = [2]{\color{brown}},
	otherkeywords       = {\&, ;, ::},
	morekeywords        = [2]{;, ::},
    linebackgroundcolor = {\ifodd\value{lstnumber}\color{istblue!10}
                           \else\color{gray!5}\fi},
    frame               = tb,
}

\definecolor{DarkBackground}{RGB}{66, 72, 78}
\definecolor{KeywordColor}{RGB}{250, 140, 213}

\lstset{
	style = myCpp
}

% Centered block with variable width
\usepackage{tikz}
\usepackage{pgfplots}

% ---------------------------------------------------------------------------- %
% Outline
% ---------------------------------------------------------------------------- %

\AtBeginSection[]{
	\frame{
		\frametitle{Outline}
		\tableofcontents[current, hideallsubsections]
	}
}

\AtBeginSubsection[]{
	\frame{
		\frametitle{Outline}
		\tableofcontents[currentsection, currentsubsection]
	}
}

% ---------------------------------------------------------------------------- %
% Cover
% ---------------------------------------------------------------------------- %

\title{The Adaptiv Framework}
\author{Nuno Alves de Sousa}
\institute{
	Instituto Superior Técnico\\
	\tiny Área Científica de Mecânica Aplicada e Aeroespacial
}
\date{\today}

\begin{document}
	
\begin{frame}[plain]
	\vspace{2.0cm}
	\centering
	\includegraphics[keepaspectratio=true, height=2cm]{figures/adaptive.png}
	\vspace{0.75cm}
	\titlepage
\end{frame}

% ---------------------------------------------------------------------------- %
% Slides
% ---------------------------------------------------------------------------- %

\section{Best coding practices}
\subsection{Software quality}

\begin{frame}{Attributes of good software}
	\begin{quote}
		\enquote{Software and cathedrals are much the same – first we build them, then we pray.}\\
		\attrib{Anonymous}
	\end{quote}

%	\begin{quote}
%		\enquote{Without requirements or design, programming is the art of adding bugs to an empty text file.}\\
%		\attrib{Louis Srygley}
%	\end{quote}
\end{frame}

\begin{frame}{Attributes of good software}
	Not \structure{\textit{what}} the program does, but \structure{\textit{how well}} it does it:
	\vspace{1\baselineskip}
	\begin{description}[\dindent Maintainability]
		\bitem[Maintainability] reduce/reverse \enquote{code entropy} \\
								cheaper/safer to change than to rewrite
		\bitem[Dependability] availability, reliability, safety, integrity
		\bitem[Efficiency] algorithmic efficiency\\
						   storage efficiency
		\bitem[Usability] \enquote{consumer} effectiveness and efficiency\\
						  elegance and clarity perceived by the user
	\end{description}
\end{frame}

\subsection{Prerequisites}
\begin{frame}{Where to start?}
	\begin{quote}
		\enquote{What happens before one gets to the coding stage is often of crucial importance to the success of the project.}
		\attrib{Meek \& Heath - Guide to Good Programming Practice}
	\end{quote}
	\vspace{1\baselineskip}
	Higher-level prerequisites to provide a solid foundation for coding:
	\begin{itemize}
		\item Coding standards
		\item Choice of programming language
		\item Life cycle, architecture, design
		\item \alert{Requirements}
	\end{itemize}
\end{frame}

\begin{frame}{Coding standards}
	Coding conventions are particularly important in collaborative projects:
	\begin{itemize}
		\item Much easier to read someone else's code
		\item Uniform style (\textit{e.g.} naming conventions for filenames, variables, etc)
		\item Deal with undereducated programmers
		\item Avoid insufficient library use
		\item Portability
		\item Commenting conventions:
			\begin{itemize}
				\itemd Speed up knowledge transfer
				\itemd Comment only what code expresses poorly (intent)
				\itemd Comments lie, code never lies
				\itemd Do not comment code modifications (use a \structure{version control system})
			\end{itemize}
	\end{itemize}
\end{frame}

\begin{frame}{Version control}
	\structure{Source code} is the most valuable asset of any software project
	\begin{block}{Version control systems (VCS)}
		\begin{itemize}
			\item Management of changes to all non-binary files
			\item Complete retrace of all versions of each file
			\item History of the authors of such changes
		\end{itemize}
	\end{block} 
	
	\begin{exampleblock}{Critical advantages}
		\begin{itemize}
			\item Rollback of all tracked changes
			\item Work in an isolated fashion
			\item Seamless team collaboration
			\item Efficient and flexible scaleability
		\end{itemize}
	\end{exampleblock}
\end{frame}

\begin{frame}{git - the world's leading version control system}
	\begin{columns}[onlytextwidth]
		\column{0.5\linewidth}
		\begin{center}
			\hspace{-1.7cm}
			\includegraphics[keepaspectratio=true, width=0.45\linewidth]{figures/git.png}
		\end{center}
		Why git?
		\begin{itemize}
			\item Free and open-source
			\item Small and fast
			\item Encourages branching
			\item Distributed
			\item Built-in IDE support
		\end{itemize}
		As a service:
		\begin{itemize}
			\item Source code hosting
			\item Code sharing platform
			\item GitHub, GitLab, etc.
		\end{itemize}
		\column{0.5\linewidth}
		\includegraphics[keepaspectratio=true, width=0.95\linewidth]{figures/github.png}
	\end{columns}
\end{frame}

\begin{frame}{Choice of programming language}
	C++ (hard, lack of knowledge, modern features)\\
	Use good well tested libraries (boost) - portability
\end{frame}

\begin{frame}{WIP}
	Life cycle, architecture, design all depend on the \alert{requirements}
\end{frame}

\subsection{Code development}
\begin{frame}{Build system}
	\begin{columns}[onlytextwidth]
		\column{0.65\linewidth}
		\begin{itemize}
			\item Open-source, cross-platform set of tools to build, test and package software.
			\item Controls compilation process using platform and compiler independent config. files
		\end{itemize}
		\column{0.30\linewidth}
		\includegraphics[keepaspectratio=true, width=1\linewidth]{figures/cmake.png}
	\end{columns}
	\vspace{1\baselineskip}
	\begin{quote}
		\centering
		\structure{The defacto standard for building C++ projects}
	\end{quote}
	\begin{exampleblock}{Advantages}
		\begin{itemize}
			\item More time for coding
			\item Supported by most popular IDEs (\textit{e.g.} VS, JetBrains, QtCreator)
			\item Support for multiple compilers (\textit{e.g.} MSVC, GCC, Clang, Intel)
			\item Easy integration of 3rd party libraries
		\end{itemize}
	\end{exampleblock}	  
\end{frame}

\begin{frame}{Testing}
	content...
\end{frame}

\begin{frame}{Benchmarking}
	content...
\end{frame}

\section{Concepts Library}

\section{Linear Algebra Library}

\begin{frame}[fragile]{Interface elegance \textit{vs} code efficiency}
	Level 3 BLAS operations $T(n) = O(n^3)$\\
	Example:
	\begin{equation}
		C \leftarrow \alpha A^T B^T + \beta C
	\end{equation}
	Desired interface:
	\lstinputlisting{code/dgemm.cpp}
\end{frame}

\begin{frame}[fragile]{Interface elegance \textit{vs} code efficiency}
	For an efficient implementation, the statement
	\lstinputlisting[linerange=12-12, firstnumber=12]{code/dgemm.cpp}
	should be translated into a call to the specialized CBLAS function:
	\lstinputlisting[linerange=12-18, firstnumber=12]{code/cblas_dgemm.cpp}
	Overhead: \\
	\begin{center}
		\textcolor{DarkGreen}{\textbf{1}} function call\\
		\textcolor{DarkGreen}{\textbf{0}} temporaries
	\end{center}
\end{frame}

\begin{frame}{Conventional operator overloading}
	Due to the normal order of evaluation of the C++ language,
	\lstinputlisting[linerange=12-12, firstnumber=12]{code/dgemm.cpp}
	
	actually generates:
	\lstinputlisting[firstline=9, firstnumber=9]{code/operator_dgemm.cpp}
\end{frame}

\begin{frame}{Conventional operator overloading}
	Overhead: \\
	\begin{center}
		\textcolor{red}{\textbf{7+}} function calls\\
		\textcolor{red}{\textbf{6}} temporaries
	\end{center}
\end{frame}

\begin{frame}{Performance comparison on a \small i7-4770k$_{(\small224\,\text{GFLOPS})}$}
	% This file was created by matlab2tikz.
%
%The latest updates can be retrieved from
%  http://www.mathworks.com/matlabcentral/fileexchange/22022-matlab2tikz-matlab2tikz
%where you can also make suggestions and rate matlab2tikz.
%
\definecolor{mycolor1}{rgb}{0.00000,0.44700,0.74100}%
\definecolor{mycolor2}{rgb}{0.85000,0.32500,0.09800}%
\pgfplotsset{compat=1.13}
%
\begin{tikzpicture}

\begin{axis}[%
	width=0.8\textwidth,
	height=0.60\textheight,
	% scaling
	scale only axis,
	% x-axis
	xmode=log,
	xmin=2,
	xmax=1e4,
	xminorticks=true,
	xlabel={Matrix dim (n x n)},
	% y-axis
	ymin=0,
	ymax=150,
	ylabel={GFLOPS},
	axis background/.style={fill=white},
	% title
	title={$C \leftarrow \alpha A^T B^T + \beta C$},
	xmajorgrids, xminorgrids,
	ymajorgrids,
	legend style={legend cell align=left, align=left, draw=white!15!black}
]
\addplot [color=mycolor1, line width=1pt]
  table[row sep=crcr]{%
2	0.0689655172413793\\
3	0.21\\
4	0.514285714285714\\
5	0.785714285714286\\
6	1.26486486486486\\
7	1.67045454545455\\
8	2.78974358974359\\
9	2.60847457627119\\
10	3.62068965517241\\
20	24.4776119402985\\
30	34.3125\\
40	42.2149837133551\\
50	61.5853658536585\\
60	91.1297071129707\\
70	105.159817351598\\
80	134.516971279373\\
90	134.258241758242\\
100	130.519480519481\\
200	42.7642103018023\\
300	45.9823857453754\\
400	49.1019432354564\\
500	50.0992977089456\\
600	50.0806182586456\\
700	51.0086712289071\\
800	50.6205043079598\\
900	51.5728406584083\\
1000	50.6090281852579\\
2000	50.2655556567595\\
3000	54.9577913091614\\
4000	50.2956464343665\\
5000	49.7921881876858\\
6000	49.6719281564627\\
7000	50.3223038047106\\
8000	50.8588853462062\\
9000	50.9636772899176\\
};
\addlegendentry{\scriptsize MKL}

\addplot [color=mycolor2, line width=1pt]
  table[row sep=crcr]{%
2	0.0384615384615385\\
3	0.108620689655172\\
4	0.205714285714286\\
5	0.352564102564103\\
6	0.563855421686747\\
7	0.735\\
8	1.00740740740741\\
9	1.24112903225806\\
10	1.48936170212766\\
20	3.78752886836028\\
30	5.73068893528184\\
40	6.75351745700886\\
50	8.2543314808761\\
60	9.92707383773929\\
70	11.4920159680639\\
80	13.3887733887734\\
90	15.175447676224\\
100	14.0245604242255\\
200	18.5502151084794\\
300	19.4610347557027\\
400	23.2769330243304\\
500	24.979038569033\\
600	27.4896522784062\\
700	26.3308491582827\\
800	26.4256829248164\\
900	30.5257431025644\\
1000	31.8238349550077\\
2000	33.6762542847584\\
3000	35.1428906066995\\
4000	37.9103354941498\\
5000	39.6672409908838\\
6000	40.8251280403682\\
7000	41.884867577567\\
8000	42.9416549954294\\
9000	43.7542874118886\\
};
\addlegendentry{\scriptsize CMatrix}

\end{axis}
\end{tikzpicture}%
\end{frame}

\end{document}